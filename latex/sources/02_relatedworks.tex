\section{Fundamentação Teórica e Trabalhos Relacionados}
Quando pensamos em roteadores geralmente temos em mente equipamentos com \textit{hardware} especializado, como ASICs(\textit{Application-Specific Integrated Circuits}). Como o nome implica, são circuítos integrados com propósito único e ser usados apenas para esta única função. Porém, as funções de um roteador não precisam ficar limitadas à um \textit{hardware} especializado, em muitos casos pode ser usado computadores convêncionais, ou seja por \textit{software} para a realização dessas tarefas. 
Os primeiros roteadores, ainda nos tempos da ARPANET(ref), eram baseados em uma arquitetura de \textit{software}. \textit{Interface Message Processor}, ou IMP, começou a ser instalado e utilizado a partir do outono de 1969.%referenciar Interface message processor for the arpa computer network. https://apps.dtic.mil/sti/pdfs/AD0686811.pdf https://apps.dtic.mil/sti/citations/AD0686811 

\textit{Software Routers} por muito tempo foram desconsiderados por não serem tão eficientes quanto roteadores ASICs porém nos dias de hoje com o advento de processadores \textit{multi-core} em até mesmo computadores de baixo custo e de \textit{multi-queue NICs} podemos ter um desempenho bem melhor que no passado usando \textit{software.(ref building a low latency linux router)}. Aleḿ disso para certas aplicações esses roteadores podem escalonar de uma forma melhor que roteadores ASICs. Porém para grande parte dos casos esses roteadores apresentam um custo mais caro comparados com roteadores de \textit{hardware}.

Para grande parte das aplicações \textit{hardware routers} são superiores nos quesitos de desempenho e até mesmo no quesito de custo. Um exemplo seria uma roteador para uso domiciliar, dificilmente seria justificavel se investir em um \textit{software router} para essa aplicação levando a pouca elasticidade necessária. Outro exemplopode ser um roteador de borda, onde o tamanho de fluxo é tão grande que uma solução de baseada em \textit{software} não seria capaz de suportar. Mas existiriam casos onde o desempenho é semelhante a um rotador ASIC? Neste trabalho será investigado se a performance de dois computadores, um de uso geral e outro de baixo custo, são capazes de competir com um roteador ASIC.
