\section{Introdução}

O roteador é um dispositivo que liga uma ou mais redes sendo sua principal função gerir o tráfego de rede através do encaminhamento dos pacotes aos endereços \ac{IP} corretos. A \textit{internet} depende de roteadores para a intercomunicação global dos \ac{AS}. 

Quando pensamos neles geralmente temos em mente equipamentos com \textit{hardware} especializado, como \acp{ASIC}. Como o nome implica, são circuítos integrados com um propósito único e que realizam apenas está função. Porém, as funções de um roteador não precisam ficar limitadas à um \textit{hardware} especializado, em muitos casos pode ser usado computadores convêncionais, ou seja por \textit{software} para a realização dessas tarefas. Neste trabalho será comparado roteadores \ac{ASIC} com \textit{software routers}.




