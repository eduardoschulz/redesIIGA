\section{Metodologia}
% TODO adicionar uma forma de comparar um hardware convencional(asics)
Neste trabalho serão apresentados quatro cenários que visam comparar o desempenho e capacidade de fluxo de dados na rede. Seram analizados roteadores \ac{ASIC} assim como roteadores rodando linux, com o intuito de comparar o desempenho da rede. 

O cenário I representará um ambiente de controle, ou seja, não possuirá roteadores. Ambos computadores se comunicaram utilizando um \textit{switch} TP-Link de 100 Megabit de banda. O cenário II e III será um ambiente com um roteador linux entre os computadores, sendo que no cenário II o roteador terá grande poder computacional comparado com o cenário III onde ele terá baixo poder computacional. Já no cenário IV será testado um roteador \ac{ASIC} rodando \textit{firmware} customizado para coleta de métricas.

Cada cenário será testado com uma série de testes de performance e estabilidade. Será também levado em conta a difença de consumo de recursos em ambos cenários. A ferramente principal desses testes será o iPerf3, que será configurado para enviar datagramas UDP de diversos tamanhos, durante um perido de 30 segundos 10 vezes.

Tem-se como hipótese que o cenário I terá o melhor desempenho comparado com o restante e que o cenário II e IV teram resultados aproximadamente parecidos. Se espera que no cenário III o desempenho será afetado pela falta de recursos do roteador.:w


\begin{table}[!h]
\centering
\label{tab:freq-conf}
\begin{tabular}{cc|}
\hline
\rowcolor[HTML]{DFDFDF} 
\multicolumn{2}{|c|}{\cellcolor[HTML]{DFDFDF}A - Thinkpad T440s}               \\ \hline
\rowcolor[HTML]{EFEFEF} 
\multicolumn{1}{|c|}{\cellcolor[HTML]{EFEFEF}Processador}         & Intel\textregistered\space Core\texttrademark\space i5 4200U          \\ \hline
\multicolumn{1}{|c|}{RAM}                    & 8GB DDR3               \\ \hline
\rowcolor[HTML]{EFEFEF} 
\multicolumn{1}{|c|}{\cellcolor[HTML]{EFEFEF}SO} & NixOS GNU/Linux                   \\ \hline
\multicolumn{1}{|c|}{\textit{kernel}}                   & 6.1.79          \\ \hline \hline
\rowcolor[HTML]{DFDFDF} 
\multicolumn{2}{|c|}{\cellcolor[HTML]{DFDFDF}B - Raspberry Pi 4}                 \\ \hline
\rowcolor[HTML]{EFEFEF} 
\multicolumn{1}{|c|}{\cellcolor[HTML]{EFEFEF}Processador} & Broadcom BCM2711 \\ \hline
\multicolumn{1}{|c|}{RAM} & 4GB LPDDR4  \\ \hline
\rowcolor[HTML]{EFEFEF} 
\multicolumn{1}{|c|}{\cellcolor[HTML]{EFEFEF}SO}                 & RPiOS Lite GNU/Linux     \\ \hline
\multicolumn{1}{|c|}{\textit{kernel}}   & 6.6.20+rpt-rpi-v8    \\ \hline 
\end{tabular}
\begin{tabular}[h]{cc|} \hline
\rowcolor[HTML]{DFDFDF} 
\multicolumn{2}{|c|}{\cellcolor[HTML]{DFDFDF}C - \textit{Custom Build}}              \\ \hline
\rowcolor[HTML]{EFEFEF} 
\multicolumn{1}{|c|}{\cellcolor[HTML]{EFEFEF}Processador} & Intel\textregistered\space Xeon\textregistered\space E5-2670v3            \\ \hline
\multicolumn{1}{|c|}{RAM}                         & 16GB DDR4 ECC              \\ \hline
\rowcolor[HTML]{EFEFEF} 
\multicolumn{1}{|c|}{\cellcolor[HTML]{EFEFEF}SO}         & NixOS GNU/Linux \\ \hline
\multicolumn{1}{|c|}{\textit{kernel}}           & 6.1.79  \\ \hline \hline
\rowcolor[HTML]{DFDFDF} 
\multicolumn{2}{|c|}{\cellcolor[HTML]{DFDFDF}D - TPLink WR741ND}                 \\ \hline
\rowcolor[HTML]{EFEFEF} 
\multicolumn{1}{|c|}{\cellcolor[HTML]{EFEFEF}Banda} & \textit{FastEthernet} 100Mbit/s         \\ \hline
\multicolumn{1}{|c|}{\textit{Firmware}}                        & OpenWRT                \\ \hline
\rowcolor[HTML]{EFEFEF} 
\multicolumn{1}{|c|}{\cellcolor[HTML]{EFEFEF}Núm. Portas}         & 4        \\ \hline
\multicolumn{1}{|c|}{Lançamento}           & 2016    \\ \hline
\end{tabular}
\caption{Especificação dos Equipamentos}
\end{table}

