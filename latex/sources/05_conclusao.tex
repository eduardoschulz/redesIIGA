\section{Conclusão e Discussões}
Depois da analise de todos esses resultados temos uma conclusão inesperada. A hipótese incial era que o roteador conseguiria acompanhar facilmente computador C, já que suas especificações ultrapassam as bandas que foram testadas nos testes. Porém, como foi visto os resultados foram muito inferiores aos cenários com \textit{Software Router} e ao controle. É preciso levar em conta que este roteador D foi lançado em 2016 porém, ele deveria ser capaz de lidar com bandas até 100Megabit/s. Como comentado na seção \ref{Metodologia} podemos confirmar que as especificações dadas pela TP-Link não condizem com os resultados coletados. 

A conclusão que podemos ter é que sim, \textit{software routers} conseguem competir com roteadores \ac{ASIC} e que talvez em certos casos, até mesmo um computador como o Raspberry Pi 4, pode ter performance superior a um \textit{hardware router} antigo. 

