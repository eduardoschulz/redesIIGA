\documentclass[12pt]{article}

\usepackage{sbc-template}
\usepackage{graphicx,url}
\usepackage[utf8]{inputenc}
\usepackage[brazil]{babel}
\usepackage{listings}
     
\sloppy
\title{Avaliação de desempenho de roteadores}
  \vspace{-0.2cm}
	\author{Eduardo Schulz\inst{1}}
  \vspace{-0.2cm}
\address{
  Universidade do Vale do Rio dos Sinos (UNISINOS), São Leopoldo, RS -- Brasil
  \vspace{-0.2cm}
	%%\email{\{schulzEduardo\}@edu.unisinos.br;}
	\email{schulzEduardo@edu.unisinos.br}
}
\begin{document} 

\maketitle

\begin{abstract}
  This meta-paper describes the style to be used in articles and short papers
  for SBC conferences. For papers in English, you should add just an abstract
  while for the papers in Portuguese, we also ask for an abstract in
  Portuguese (``resumo''). In both cases, abstracts should not have more than
  10 lines and must be in the first page of the paper.
\end{abstract}
     
\begin{resumo} 
  Este meta-artigo descreve o estilo a ser usado na confecção de artigos e
  resumos de artigos para publicação nos anais das conferências organizadas
  pela SBC. É solicitada a escrita de resumo e abstract apenas para os artigos
  escritos em português. Artigos em inglês deverão apresentar apenas abstract.
  Nos dois casos, o autor deve tomar cuidado para que o resumo (e o abstract)
  não ultrapassem 10 linhas cada, sendo que ambos devem estar na primeira
  página do artigo.
\end{resumo}


\section{Introdução}


\section{Fundamentação Teórica e Trabalhos Relacionados}

Quando pensamos em roteadores geralmente temos em mente os equipamentos com \textit{hardware} especializado, como ASICs(\textit{Application-Specific Integrated Circuits}). Como o nome implica, são CIs com propósito único e podem ser usados para esta única função. Porém, as funções de um roteador não precisam ficar limitadas à um \textit{hardware} especializado, em muitos casos pode ser usado computadores convêncionais para a realização dessas funcionalidades.

falar sobre trab [1] e [2]

falar sobre outro trabalho.

%%TODO adicionar referencias aos artigos de performance e comparacao entre asics e cpus.

\section{Metodologia}
% TODO adicionar uma forma de comparar um hardware convencional(asics)
Foram definidos dois cenários principais que devem ser testados durante o trabalho:
\begin{enumerate}
	\item[A] Uma rede com dois \textit{hosts} conectados na mesma rede.
	\item[B] Duas redes seperadas por um roteador com cada \textit{host} em redes diferentes. 
\end{enumerate}

Cada cenário será testado com uma série de testes de performance e estabilidade. Será também levado em conta a difença de consumo de recursos em ambos cenários. A ferramente principal desses testes será o iPerf3, que será configurado para enviar datagramas UDP de diversos tamanhos, durante um perido de 30 segundos 10 vezes.

Também será testado o roteador com máquinas com recursos mais limitados e com máquinas com mais recursos computacionais. Os resultados desses testes devem ser comparados entre si e com valores que a rede teóricamente poderia atingir.

Tem-se como hipótese que o cenário B terá um desempenho menor, principalmente quando o roteador possuir menos recursos computacionais. É esperado que essas operações que estaram sendo feitas por \textit{software} teram um desempenho menor que em \textit{hardware} especializado.


\section{Avaliação de desempenho e funcionalidade}
Nessa sessão será apresentado o os ambientes dos cenários de testes, principalmente como foram configurados e seus respectivos \textit{hardwares}. Também será exposto os resultados dos experimentos realizados neste trabalho.
\subsection{Equipamentos}
Como explicado na ??botar secao dos cenarios??, o testes serão realizados seguindo tais cenários com diversos equipamentos, que devem ser alterados dependendo do cenário selecionado.
A coleta de métricas de consumo de processador, memória, entrada e saída de \textit{bits} pela interface de rede
foram realizadas usando o \textit{software} Prometheus em conjunto com o Grafana para a visualização gráfica dos dados.

%Para a realização dos cenários de testes foram criados \textit{testbeds} compostos por três computadores, um roteador e um \textit{switch}. 
%\subsection{\textit{Testbed}}
%Para a realização dos dois cenários de testes foi criado uma \textit{Testbed} composta por três máquinas rodando GNU/Linux.
%Todas as máquinas estam equipadas com interfaces de redes \textit{Gigabit} para maior banda de tranfêrencia. 

\subsubsection{\textit{Host} A}
O \textit{Host} A foi definido como um \textit{laptop} Thinkpad T440s com um Intel\textregistered\space Core\texttrademark\space i5-4200U. Esta máquina está rodando NixOS com o \textit{kernel} Linux 6.1.79. %%Apenas foi necessário adicionar uma rota até o \textit{Host} B através do roteador para que o sistema funcione no cenário 2:

\subsubsection{\textit{Host} B}
O \textit{Host} B é um Raspberry Pi 3B. Este \textit{Single Board Computer} está rodando o Raspberry Pi OS, baseado na distribuição Debian Linux, com o \textit{kernel} Linux 6.1.21-v8+. %Como citado acima, este \textit{Host} tomara o \textit{hardware} do roteador e vice-versa. Também foi necessário adicionar uma rota, até o \textit{Host} A através do roteador para que o sistema funcione no cenário 2:

\subsubsection{\textit{Host} C}
Este computar está equipado com um Intel\textregistered\space Xeon\textregistered\space E5-2670 v3 com 12 núcleos e 24 \textit{threads}.
O sistema operacional foi GNU/Linux, especificamente a distribuição NixOS usando o \textit{Linux 6.1.79}. Sendo o \textit{host} com as melhores especificações, ele também foi escolhido para hospedar o Grafana e Prometheus.


\subsubsection{\textit{Host} D}
Para este \textit{host} foi escolhido um TP-Link\textregistered\space TL-WR741ND com 5 portas. Este roteador foi configurado para executar o \textit{firmware} customizado OpenWRT, baseado no GNU/Linux. Com acesso privelegiado ao \textit{hardware} do roteador poderemos coletar muito mais informações do que um roteador COTS(\textit{Commercial-Off-The-Shelf}). Também poderemos comparar a performance entre um roteador \textbf{VSR} e \textbf{PHR}.

Este \textit{setup} possuí duas interfaces de rede \textit{Gigabit} sendo uma delas destinadas a função, equivalente a um roteador convencional, WAN e a outra interface se destina a função LAN. 
A interface WAN foi configurada para ganhar um endereço IP utilizando DHCP.
Já a interface LAN foi configurada de forma manual definindo a mascara de rede /24, um endereço IP estático e um \textit{gateway} na rede WAN.
Para a configuração de rede foram utilizados os seguintes comandos:

\begin{lstlisting}
sudo ip route add 192.168.0.0/24 via 10.0.1.1 dev eth0
\end{lstlisting}

\begin{lstlisting}
sudo systemctl net.ipv4.ip_forward 
sudo ip addr add 10.0.1.1/24 dev enp7s0 
sudo iptables -t nat -A POSTROUTING -o enp9s0 -j MASQUERADE 
\end{lstlisting}



\subsection{Resultados}

\section{Conclusão e Discussões}
\bibliographystyle{sbc}
\bibliography{sbc-template}

\end{document}
