\documentclass[12pt]{article}

\usepackage{sbc-template}
\usepackage{acronym}
\usepackage{graphicx,url}
\usepackage[utf8]{inputenc}
\usepackage[brazil]{babel}
\usepackage{listings}
\usepackage{multirow}
\usepackage{makecell} 
\usepackage{tabularx}
\usepackage{multirow,array}
\usepackage[table,xcdraw]{xcolor}
\usepackage{float}

\usepackage{acronym} 
\acrodef{3GPP}[3GPP]{\textit{3rd Generation Partnership Project}}
\acrodef{AI}[AI]{Artificial Intelligence}
\acrodef{B5G}[B5G]{Beyond 5G}
\acrodef{CL}[CL]{Central layer}
\acrodef{CN}[CN]{Cloud server node}
\acrodef{CNF}[CNF]{Cloud-native Network Function}
\acrodef{COTS}[COTS]{Commercial Off-The-Shelf}
\acrodef{Dis}[Dis]{Disaggregated}
\acrodef{Dis-RIC}[Dis-RIC]{Disaggregated near-RT \ac{RIC}}
\acrodef{DU}[DU]{Distributed Unit}
\acrodef{E2AP}[E2AP]{E2 Application Protocol}
\acrodef{E2T}[E2T]{E2 Termination}
\acrodef{E2N}[E2N]{E2 Node}
\acrodef{E2Sim}[E2Sim]{E2 Simulator}
\acrodef{eNB}[eNB]{Evolved Node B}
\acrodef{EPC}[EPC]{Evolved Packet Core}
\acrodef{gNB}[gNB]{Next Generation Node B}
\acrodef{IL}[IL]{Internal layer}
\acrodef{KPI}[KPI]{Key Performance Indicator}
\acrodef{K8s}[K8s]{Kubernetes}
\acrodef{LTE}[LTE]{Long-term Evolution}
\acrodef{MC}[MC]{Monolithic Co-located}
\acrodef{MD}[MD]{Monolithic Distributed}
\acrodef{MD-RIC}[MD-RIC]{Monolithic Distributed near-RT \ac{RIC}}
\acrodef{MEC}[MEC]{Multi-access Edge Computing}
\acrodef{MC-RIC}[MC-RIC]{Monolithic Co-located near-RT \ac{RIC}}
\acrodef{ML}[ML]{Metropolitan layer}
\acrodef{NIB}[NIB]{Network Information Base}
\acrodef{NFV}[NFV]{Network Function Virtualization}
\acrodef{Non-RT RIC}[non-RT RIC]{Non Real-Time \ac{RIC}}
\acrodef{Near-RT RIC}[near-RT RIC]{Near Real-Time \ac{RIC}}
\acrodef{O-Cloud}[O-Cloud]{Open Cloud}
\acrodef{O-RAN}[O-RAN]{\textit{Open Radio Access Network Alliance}}
\acrodef{O-RAN SC}[OSC]{\ac{O-RAN} Software Community}
\acrodef{RMR}[RMR]{\ac{RIC} Message Router}
\acrodef{RIC}[RIC]{RAN Intelligent Controller}
\acrodef{RIC-O}[RIC-O]{\ac{RIC} Orchestrator}
\acrodef{RAN}[RAN]{\textit{Radio Access Network}}
\acrodef{SCTP}[SCTP]{Stream Control Transmission Protocol}
\acrodef{SDL}[SDL]{Shared Data Layer}
\acrodef{SDN}[SDN]{Software-Defined Networking}
\acrodef{SDR}[SDR]{Software Defined Radio}
\acrodef{SMO}[SMO]{Service Management and Orchestration}
\acrodef{srsRAN}[srsRAN]{Software Radio Systems \ac{RAN}}
\acrodef{STSL}[STSL]{Shared Time-Series Layer}
\acrodef{TDR}[TDR]{Time to Disaster Recovery}
\acrodef{UE}[UE]{User Equipment}
\acrodef{USRP}[USRP]{Universal Software Radio Peripheral}
\acrodef{VM}[VM]{Virtual Machine}
\acrodef{VNF}[VNF]{Virtual Network Function}
     
\sloppy
\title{Avaliação de desempenho de roteadores}
  \vspace{-0.2cm}
	\author{Eduardo Schulz\inst{1}}
  \vspace{-0.2cm}
\address{
  Universidade do Vale do Rio dos Sinos (UNISINOS), São Leopoldo, RS -- Brasil
  \vspace{-0.2cm}
	%%\email{\{schulzEduardo\}@edu.unisinos.br;}
	\email{schulzEduardo@edu.unisinos.br}
}
\begin{document} 

\maketitle

\begin{abstract}
  This meta-paper describes the style to be used in articles and short papers
  for SBC conferences. For papers in English, you should add just an abstract
  while for the papers in Portuguese, we also ask for an abstract in
  Portuguese (``resumo''). In both cases, abstracts should not have more than
  10 lines and must be in the first page of the paper.
\end{abstract}
     
\begin{resumo} 
  Este meta-artigo descreve o estilo a ser usado na confecção de artigos e
  resumos de artigos para publicação nos anais das conferências organizadas
  pela SBC. É solicitada a escrita de resumo e abstract apenas para os artigos
  escritos em português. Artigos em inglês deverão apresentar apenas abstract.
  Nos dois casos, o autor deve tomar cuidado para que o resumo (e o abstract)
  não ultrapassem 10 linhas cada, sendo que ambos devem estar na primeira
  página do artigo.
\end{resumo}


\section{Introdução}

Quando pensamos em roteadores geralmente temos em mente equipamentos com \textit{hardware} especializado, como \acp{ASIC}. Como o nome implica, são circuítos integrados com propósito único (alterar) ser usados apenas para esta única função. Porém, as funções de um roteador não precisam ficar limitadas à um \textit{hardware} especializado, em muitos casos pode ser usado computadores convêncionais, ou seja por \textit{software} para a realização dessas tarefas. 

Quando pensamos em roteadores geralmente temos em mente equipamentos com \textit{hardware} especializado, como ASICs(\textit{Application-Specific Integrated Circuits}). Como o nome implica, são circuítos integrados com propósito único e ser usados apenas para esta única função. Porém, as funções de um roteador não precisam ficar limitadas à um \textit{hardware} especializado, em muitos casos pode ser usado computadores convêncionais, ou seja por \textit{software} para a realização dessas tarefas. 
Os primeiros roteadores, ainda nos tempos da ARPANET(ref), eram baseados em uma arquitetura de \textit{software}. \textit{Interface Message Processor}, ou IMP, começou a ser instalado e utilizado a partir do outono de 1969.%referenciar Interface message processor for the arpa computer network. https://apps.dtic.mil/sti/pdfs/AD0686811.pdf https://apps.dtic.mil/sti/citations/AD0686811 

\textit{Software Routers} por muito tempo foram desconsiderados por não serem tão eficientes quanto roteadores ASICs porém nos dias de hoje com o advento de processadores \textit{multi-core} em até mesmo computadores de baixo custo e de \textit{multi-queue NICs} podemos ter um desempenho bem melhor que no passado usando \textit{software.(ref building a low latency linux router)}. Aleḿ disso para certas aplicações esses roteadores podem escalonar de uma forma melhor que roteadores ASICs. Porém para grande parte dos casos esses roteadores apresentam um custo mais caro comparados com roteadores de \textit{hardware}.


Para grande parte das aplicações \textit{hardware routers} são superiores nos quesitos de desempenho e até mesmo no quesito de custo. Um exemplo seria uma roteador para uso domiciliar, dificilmente seria justificavel se investir em um \textit{software router} para essa aplicação levando a pouca elasticidade necessária. Outro exemplopode ser um roteador de borda, onde o tamanho de fluxo é tão grande que uma solução de baseada em \textit{software} não seria capaz de suportar. Mas existiriam casos onde o desempenho é semelhante a um rotador ASIC? Neste trabalho será investigado se a performance de dois computadores, um de uso geral e outro de baixo custo, são capazes de competir com um roteador ASIC.

\section{Metodologia}
% TODO adicionar uma forma de comparar um hardware convencional(asics)
Neste trabalho serão apresentados quatro cenários que visam comparar o desempenho e capacidade de fluxo de dados na rede. Seram analizados roteadores \ac{ASIC} assim como roteadores rodando linux, com o intuito de comparar o desempenho da rede. 

O cenário I representará um ambiente de controle, ou seja, não possuirá roteadores. Ambos computadores se comunicaram utilizando um \textit{switch} TP-Link de 100 Megabit de banda. O cenário II e III será um ambiente com um roteador linux entre os computadores, sendo que no cenário II o roteador terá grande poder computacional comparado com o cenário III onde ele terá baixo poder computacional. Já no cenário IV será testado um roteador \ac{ASIC} rodando \textit{firmware} customizado para coleta de métricas.

Cada cenário será testado com uma série de testes de performance e estabilidade. Será também levado em conta a difença de consumo de recursos em ambos cenários. A ferramente principal desses testes será o iPerf3, que será configurado para enviar datagramas UDP de diversos tamanhos, durante um perido de 30 segundos 10 vezes.

Tem-se como hipótese que o cenário I terá o melhor desempenho comparado com o restante e que o cenário II e IV teram resultados aproximadamente parecidos. Se espera que no cenário III o desempenho será afetado pela falta de recursos do roteador.:w


\section{Avaliação de desempenho e funcionalidade}
Nessa sessão será apresentado o os ambientes dos cenários de testes, principalmente como foram configurados e seus respectivos \textit{hardwares}. Também será exposto os resultados dos experimentos realizados neste trabalho.
\subsection{Equipamentos}

Como explicado na seção 3, o testes serão realizados seguindo cenários com diversos equipamentos, que devem ser alterados dependendo do cenário selecionado.
A coleta de métricas de consumo de processador, memória, entrada e saída de \textit{bits} pela interface de rede
foram realizadas usando o \textit{software} Prometheus em conjunto com o Grafana para a visualização gráfica dos dados.


\begin{table}[!h]
\centering
\label{tab:freq-conf}
\begin{tabular}{cc|}
\hline
\rowcolor[HTML]{DFDFDF} 
\multicolumn{2}{|c|}{\cellcolor[HTML]{DFDFDF}A - Thinkpad T440s}               \\ \hline
\rowcolor[HTML]{EFEFEF} 
\multicolumn{1}{|c|}{\cellcolor[HTML]{EFEFEF}Processador}         & Intel\textregistered\space Core\texttrademark\space i5 4200U          \\ \hline
\multicolumn{1}{|c|}{RAM}                    & 8GB DDR3               \\ \hline
\rowcolor[HTML]{EFEFEF} 
\multicolumn{1}{|c|}{\cellcolor[HTML]{EFEFEF}SO} & NixOS GNU/Linux                   \\ \hline
\multicolumn{1}{|c|}{\textit{kernel}}                   & 6.1.79          \\ \hline \hline
\rowcolor[HTML]{DFDFDF} 
\multicolumn{2}{|c|}{\cellcolor[HTML]{DFDFDF}B - Raspberry Pi 4}                 \\ \hline
\rowcolor[HTML]{EFEFEF} 
\multicolumn{1}{|c|}{\cellcolor[HTML]{EFEFEF}Processador} & Broadcom BCM2711 \\ \hline
\multicolumn{1}{|c|}{RAM} & 4GB LPDDR4  \\ \hline
\rowcolor[HTML]{EFEFEF} 
\multicolumn{1}{|c|}{\cellcolor[HTML]{EFEFEF}SO}                 & RPiOS Lite GNU/Linux     \\ \hline
\multicolumn{1}{|c|}{\textit{kernel}}   & 6.6.20+rpt-rpi-v8    \\ \hline 
\end{tabular}
\begin{tabular}[h]{cc|} \hline
\rowcolor[HTML]{DFDFDF} 
\multicolumn{2}{|c|}{\cellcolor[HTML]{DFDFDF}C - \textit{Custom Build}}              \\ \hline
\rowcolor[HTML]{EFEFEF} 
\multicolumn{1}{|c|}{\cellcolor[HTML]{EFEFEF}Processador} & Intel\textregistered\space Xeon\textregistered\space E5-2670v3            \\ \hline
\multicolumn{1}{|c|}{RAM}                         & 16GB DDR4 ECC              \\ \hline
\rowcolor[HTML]{EFEFEF} 
\multicolumn{1}{|c|}{\cellcolor[HTML]{EFEFEF}SO}         & NixOS GNU/Linux \\ \hline
\multicolumn{1}{|c|}{\textit{kernel}}           & 6.1.79  \\ \hline \hline
\rowcolor[HTML]{DFDFDF} 
\multicolumn{2}{|c|}{\cellcolor[HTML]{DFDFDF}D - TPLink WR741ND}                 \\ \hline
\rowcolor[HTML]{EFEFEF} 
\multicolumn{1}{|c|}{\cellcolor[HTML]{EFEFEF}Banda} & \textit{FastEthernet} 100Mbit/s         \\ \hline
\multicolumn{1}{|c|}{\textit{Firmware}}                        & OpenWRT                \\ \hline
\rowcolor[HTML]{EFEFEF} 
\multicolumn{1}{|c|}{\cellcolor[HTML]{EFEFEF}Núm. Portas}         & 4        \\ \hline
\multicolumn{1}{|c|}{Lançamento}           & 2016    \\ \hline
\end{tabular}
\caption{Especificação dos Equipamentos}
\end{table}

%Para a realização dos cenários de testes foram criados \textit{testbeds} compostos por três computadores, um roteador e um \textit{switch}. 
%\subsection{\textit{Testbed}}
%Para a realização dos dois cenários de testes foi criado uma \textit{Testbed} composta por três máquinas rodando GNU/Linux.
%Todas as máquinas estam equipadas com interfaces de redes \textit{Gigabit} para maior banda de tranfêrencia. 


\subsubsection{Configuração dos Equipamentos}
No cenário I os equipamentos A e B foram configurados da seguinte forma:

\begin{lstlisting}
sudo ip addr add 10.0.1.2/24 dev {iface} # A
sudo ip addr add 10.0.1.3/24 dev {iface} # B
\end{lstlisting}

Nos cenários II e III, o roteador recebeu um endereço \ac{IP} fixo assim como o segundo computador. O primeiro computador recebeu um endereço \ac{IP} dinâmico dado pela rede acima do roteador.

Para o roteador: 

\begin{lstlisting}
sudo systemctl net.ipv4.ip_forward 
sudo ip addr add 10.0.1.1/24 dev enp7s0 
sudo iptables -t nat -A POSTROUTING -o enp9s0 -j MASQUERADE 
\end{lstlisting}

Para o segundo computador:
\begin{lstlisting}
sudo ip addr add 10.0.1.2/24 dev {iface}
sudo ip route add 192.168.0.0/24 via 10.0.1.1 dev {iface}
\end{lstlisting}





\section{Conclusão e Discussões}
Depois da analise de todos esses resultados temos uma conclusão inesperada. A hipótese incial era que o roteador conseguiria acompanhar facilmente computador C, já que suas especificações ultrapassam as bandas que foram testadas nos testes. Porém, como foi visto os resultados foram muito inferiores aos cenários com \textit{Software Router} e ao controle. É preciso levar em conta que este roteador D foi lançado em 2016 porém, ele deveria ser capaz de lidar com bandas até 100Megabit/s. Como comentado na seção \ref{Metodologia} podemos confirmar que as especificações dadas pela TP-Link não condizem com os resultados coletados. 

A conclusão que podemos ter é que sim, \textit{software routers} conseguem competir com roteadores \ac{ASIC} e que talvez em certos casos, até mesmo um computador como o Raspberry Pi 4, pode ter performance superior a um \textit{hardware router} antigo. 



\bibliographystyle{sbc}
\bibliography{sources/refs}

\end{document}
